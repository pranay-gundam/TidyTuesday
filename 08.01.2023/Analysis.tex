% Options for packages loaded elsewhere
\PassOptionsToPackage{unicode}{hyperref}
\PassOptionsToPackage{hyphens}{url}
%
\documentclass[
]{article}
\usepackage{amsmath,amssymb}
\usepackage{lmodern}
\usepackage{iftex}
\ifPDFTeX
  \usepackage[T1]{fontenc}
  \usepackage[utf8]{inputenc}
  \usepackage{textcomp} % provide euro and other symbols
\else % if luatex or xetex
  \usepackage{unicode-math}
  \defaultfontfeatures{Scale=MatchLowercase}
  \defaultfontfeatures[\rmfamily]{Ligatures=TeX,Scale=1}
\fi
% Use upquote if available, for straight quotes in verbatim environments
\IfFileExists{upquote.sty}{\usepackage{upquote}}{}
\IfFileExists{microtype.sty}{% use microtype if available
  \usepackage[]{microtype}
  \UseMicrotypeSet[protrusion]{basicmath} % disable protrusion for tt fonts
}{}
\makeatletter
\@ifundefined{KOMAClassName}{% if non-KOMA class
  \IfFileExists{parskip.sty}{%
    \usepackage{parskip}
  }{% else
    \setlength{\parindent}{0pt}
    \setlength{\parskip}{6pt plus 2pt minus 1pt}}
}{% if KOMA class
  \KOMAoptions{parskip=half}}
\makeatother
\usepackage{xcolor}
\usepackage[margin=1in]{geometry}
\usepackage{color}
\usepackage{fancyvrb}
\newcommand{\VerbBar}{|}
\newcommand{\VERB}{\Verb[commandchars=\\\{\}]}
\DefineVerbatimEnvironment{Highlighting}{Verbatim}{commandchars=\\\{\}}
% Add ',fontsize=\small' for more characters per line
\usepackage{framed}
\definecolor{shadecolor}{RGB}{248,248,248}
\newenvironment{Shaded}{\begin{snugshade}}{\end{snugshade}}
\newcommand{\AlertTok}[1]{\textcolor[rgb]{0.94,0.16,0.16}{#1}}
\newcommand{\AnnotationTok}[1]{\textcolor[rgb]{0.56,0.35,0.01}{\textbf{\textit{#1}}}}
\newcommand{\AttributeTok}[1]{\textcolor[rgb]{0.77,0.63,0.00}{#1}}
\newcommand{\BaseNTok}[1]{\textcolor[rgb]{0.00,0.00,0.81}{#1}}
\newcommand{\BuiltInTok}[1]{#1}
\newcommand{\CharTok}[1]{\textcolor[rgb]{0.31,0.60,0.02}{#1}}
\newcommand{\CommentTok}[1]{\textcolor[rgb]{0.56,0.35,0.01}{\textit{#1}}}
\newcommand{\CommentVarTok}[1]{\textcolor[rgb]{0.56,0.35,0.01}{\textbf{\textit{#1}}}}
\newcommand{\ConstantTok}[1]{\textcolor[rgb]{0.00,0.00,0.00}{#1}}
\newcommand{\ControlFlowTok}[1]{\textcolor[rgb]{0.13,0.29,0.53}{\textbf{#1}}}
\newcommand{\DataTypeTok}[1]{\textcolor[rgb]{0.13,0.29,0.53}{#1}}
\newcommand{\DecValTok}[1]{\textcolor[rgb]{0.00,0.00,0.81}{#1}}
\newcommand{\DocumentationTok}[1]{\textcolor[rgb]{0.56,0.35,0.01}{\textbf{\textit{#1}}}}
\newcommand{\ErrorTok}[1]{\textcolor[rgb]{0.64,0.00,0.00}{\textbf{#1}}}
\newcommand{\ExtensionTok}[1]{#1}
\newcommand{\FloatTok}[1]{\textcolor[rgb]{0.00,0.00,0.81}{#1}}
\newcommand{\FunctionTok}[1]{\textcolor[rgb]{0.00,0.00,0.00}{#1}}
\newcommand{\ImportTok}[1]{#1}
\newcommand{\InformationTok}[1]{\textcolor[rgb]{0.56,0.35,0.01}{\textbf{\textit{#1}}}}
\newcommand{\KeywordTok}[1]{\textcolor[rgb]{0.13,0.29,0.53}{\textbf{#1}}}
\newcommand{\NormalTok}[1]{#1}
\newcommand{\OperatorTok}[1]{\textcolor[rgb]{0.81,0.36,0.00}{\textbf{#1}}}
\newcommand{\OtherTok}[1]{\textcolor[rgb]{0.56,0.35,0.01}{#1}}
\newcommand{\PreprocessorTok}[1]{\textcolor[rgb]{0.56,0.35,0.01}{\textit{#1}}}
\newcommand{\RegionMarkerTok}[1]{#1}
\newcommand{\SpecialCharTok}[1]{\textcolor[rgb]{0.00,0.00,0.00}{#1}}
\newcommand{\SpecialStringTok}[1]{\textcolor[rgb]{0.31,0.60,0.02}{#1}}
\newcommand{\StringTok}[1]{\textcolor[rgb]{0.31,0.60,0.02}{#1}}
\newcommand{\VariableTok}[1]{\textcolor[rgb]{0.00,0.00,0.00}{#1}}
\newcommand{\VerbatimStringTok}[1]{\textcolor[rgb]{0.31,0.60,0.02}{#1}}
\newcommand{\WarningTok}[1]{\textcolor[rgb]{0.56,0.35,0.01}{\textbf{\textit{#1}}}}
\usepackage{graphicx}
\makeatletter
\def\maxwidth{\ifdim\Gin@nat@width>\linewidth\linewidth\else\Gin@nat@width\fi}
\def\maxheight{\ifdim\Gin@nat@height>\textheight\textheight\else\Gin@nat@height\fi}
\makeatother
% Scale images if necessary, so that they will not overflow the page
% margins by default, and it is still possible to overwrite the defaults
% using explicit options in \includegraphics[width, height, ...]{}
\setkeys{Gin}{width=\maxwidth,height=\maxheight,keepaspectratio}
% Set default figure placement to htbp
\makeatletter
\def\fps@figure{htbp}
\makeatother
\setlength{\emergencystretch}{3em} % prevent overfull lines
\providecommand{\tightlist}{%
  \setlength{\itemsep}{0pt}\setlength{\parskip}{0pt}}
\setcounter{secnumdepth}{-\maxdimen} % remove section numbering
\ifLuaTeX
  \usepackage{selnolig}  % disable illegal ligatures
\fi
\IfFileExists{bookmark.sty}{\usepackage{bookmark}}{\usepackage{hyperref}}
\IfFileExists{xurl.sty}{\usepackage{xurl}}{} % add URL line breaks if available
\urlstyle{same} % disable monospaced font for URLs
\hypersetup{
  pdftitle={Analysis},
  pdfauthor={Pranay Gundam},
  hidelinks,
  pdfcreator={LaTeX via pandoc}}

\title{Analysis}
\author{Pranay Gundam}
\date{2023-08-07}

\begin{document}
\maketitle

\hypertarget{state-name-etymology}{%
\section{State Name Etymology}\label{state-name-etymology}}

First as some color commentary for this first TidyTuesday analysis. This
data set doesn't give one much to work with. There are some interesting
visualizations one can make, but I found it quite difficult to ask any
meaningful well conditioned research questions. That being said the best
I could think of for now was: The effects of native american vs european
influence on states.

The influence of Native American vs European culture is not a very
concrete term. In the context of this project, we will be using the
etymology of a state's name as a proxy for Native American vs European
influence on a state. This is of course not anywhere near perfectly
analogous but gives some level of insight about the early formation of a
given state.

\begin{Shaded}
\begin{Highlighting}[]
\NormalTok{states }\OtherTok{=} \FunctionTok{read\_csv}\NormalTok{(}\StringTok{"states.csv"}\NormalTok{)}
\end{Highlighting}
\end{Shaded}

\begin{verbatim}
## Rows: 50 Columns: 14
## -- Column specification --------------------------------------------------------
## Delimiter: ","
## chr  (5): state, postal_abbreviation, capital_city, largest_city, demonym
## dbl  (8): population_2020, total_area_mi2, total_area_km2, land_area_mi2, la...
## date (1): admission
## 
## i Use `spec()` to retrieve the full column specification for this data.
## i Specify the column types or set `show_col_types = FALSE` to quiet this message.
\end{verbatim}

\begin{Shaded}
\begin{Highlighting}[]
\NormalTok{etymo }\OtherTok{=} \FunctionTok{read\_csv}\NormalTok{(}\StringTok{"state\_name\_etymology.csv"}\NormalTok{)}
\end{Highlighting}
\end{Shaded}

\begin{verbatim}
## Rows: 56 Columns: 5
## -- Column specification --------------------------------------------------------
## Delimiter: ","
## chr  (4): state, language, words_in_original, meaning
## date (1): date_named
## 
## i Use `spec()` to retrieve the full column specification for this data.
## i Specify the column types or set `show_col_types = FALSE` to quiet this message.
\end{verbatim}

\hypertarget{eda}{%
\subsection{EDA}\label{eda}}

There are originally two datasets that

\begin{Shaded}
\begin{Highlighting}[]
\NormalTok{mergedf }\OtherTok{=} \FunctionTok{merge}\NormalTok{(states, etymo, }\AttributeTok{by =} \StringTok{"state"}\NormalTok{)}
\FunctionTok{head}\NormalTok{(mergedf)}
\end{Highlighting}
\end{Shaded}

\begin{verbatim}
##        state postal_abbreviation capital_city largest_city  admission
## 1    Alabama                  AL   Montgomery   Huntsville 1819-12-14
## 2     Alaska                  AK       Juneau    Anchorage 1959-01-03
## 3    Arizona                  AZ      Phoenix      Phoenix 1912-02-14
## 4    Arizona                  AZ      Phoenix      Phoenix 1912-02-14
## 5   Arkansas                  AR  Little Rock  Little Rock 1836-06-15
## 6 California                  CA   Sacramento  Los Angeles 1850-09-09
##   population_2020 total_area_mi2 total_area_km2 land_area_mi2 land_area_km2
## 1         5024279          52420         135767         50645        131171
## 2          733391         665384        1723337        570641       1477953
## 3         7151502         113990         295234        113594        294207
## 4         7151502         113990         295234        113594        294207
## 5         3011524          53179         137732         52035        134771
## 6        39538223         163695         423967        155779        403466
##   water_area_mi2 water_area_km2 n_representatives     demonym date_named
## 1           1775           4597                 7   Alabamian 1692-04-19
## 2          94743         245384                 1     Alaskan 1666-12-02
## 3            396           1026                 9    Arizonan 1883-02-01
## 4            396           1026                 9    Arizonan 1883-02-01
## 5           1143           2961                 4    Arkansan 1796-07-20
## 6           7916          20501                52 Californian 1850-05-22
##                                      language             words_in_original
## 1                             Choctaw/Alabama            alba amo/Albaamaha
## 2                           Aleut via Russian alaxsxaq via Аляска (Alyaska)
## 3                                      Basque                     aritz ona
## 4                         Oʼodham via Spanish       ali ṣona-g via Arizonac
## 5 Kansa, Quapaw via Miami-Illinois and French          akakaze via Arcansas
## 6                                     Spanish                    california
##                                                                                                                                                                                                                                                                                                                                                                                                                        meaning
## 1                                                                                                                                                                                                                                                                               'Thicket-clearers' or 'plant-cutters', from alba, '(medicinal) plants', and amo, 'to clear'. The modern Choctaw name for the tribe is Albaamu.
## 2                                                                                                                                                                                                                                                                                                                                         'Mainland' (literally 'the object towards which the action of the sea is directed').
## 3                                                                                                                                                                                                                                                                                                                                                                                                              'The good oak'.
## 4                                                                                                                                                                                                                                                                                                                                                                                                    'Having a little spring'.
## 5                                                                                                                                                                                                                                                       Borrowed from a French spelling of a Miami-Illinois rendering of the tribal name kką:ze (see Kansas, below), which the Miami and Illinois used to refer to the Quapaw.
## 6 Probably named for the fictional Island of California ruled by Queen Calafia in the 16th-century novel Las sergas de Esplandián by Garci Rodríguez de Montalvo..mw-parser-output .hatnote{font-style:italic}.mw-parser-output div.hatnote{padding-left:1.6em;margin-bottom:0.5em}.mw-parser-output .hatnote i{font-style:normal}.mw-parser-output .hatnote+link+.hatnote{margin-top:-0.5em}See also: Etymology of California
\end{verbatim}

\begin{Shaded}
\begin{Highlighting}[]
\FunctionTok{unique}\NormalTok{(mergedf}\SpecialCharTok{$}\NormalTok{language)}
\end{Highlighting}
\end{Shaded}

\begin{verbatim}
##  [1] "Choctaw/Alabama"                             
##  [2] "Aleut via Russian"                           
##  [3] "Basque"                                      
##  [4] "Oʼodham via Spanish"                         
##  [5] "Kansa, Quapaw via Miami-Illinois and French" 
##  [6] "Spanish"                                     
##  [7] "Eastern Algonquian, Mohegan-Pequot"          
##  [8] "French via English"                          
##  [9] "Latin via English (ultimately from Greek)"   
## [10] "Hawaiian"                                    
## [11] "Germanic"                                    
## [12] "Plains Apache"                               
## [13] "Algonquian, Miami-Illinois via French"       
## [14] "Latin (ultimately from Proto-Indo-Iranian)"  
## [15] "Dakota, Chiwere via French"                  
## [16] "Kansa via French"                            
## [17] "Iroquoian"                                   
## [18] "French (ultimately from Frankish)"           
## [19] "English"                                     
## [20] "French"                                      
## [21] "English (ultimately from Hebrew)"            
## [22] "Eastern Algonquian, Massachusett"            
## [23] "Ojibwe via French"                           
## [24] "Dakota"                                      
## [25] "Miami-Illinois via French"                   
## [26] "Chiwere"                                     
## [27] "English (ultimately from Old English)"       
## [28] "English (ultimately from Old Norse)"         
## [29] "Nahuatl via Spanish"                         
## [30] "Latin via English (ultimately from Frankish)"
## [31] "Sioux/Dakota"                                
## [32] "Seneca via French"                           
## [33] "Choctaw"                                     
## [34] "Unknown"                                     
## [35] "Welsh and Latin"                             
## [36] "Dutch"                                       
## [37] "Greek"                                       
## [38] "Cherokee"                                    
## [39] "Caddo via Spanish"                           
## [40] "Apache via Spanish"                          
## [41] "Ute via Spanish"                             
## [42] "Latin"                                       
## [43] "Munsee/Delaware"
\end{verbatim}

\begin{Shaded}
\begin{Highlighting}[]
\NormalTok{mergedf }\SpecialCharTok{\%\textgreater{}\%}
  \FunctionTok{ggplot}\NormalTok{(}\FunctionTok{aes}\NormalTok{(}\AttributeTok{x =}\NormalTok{ date\_named, }\AttributeTok{y =}\NormalTok{ population\_2020)) }\SpecialCharTok{+}
  \FunctionTok{geom\_point}\NormalTok{() }\SpecialCharTok{+}
  \FunctionTok{geom\_smooth}\NormalTok{(}\AttributeTok{se =} \ConstantTok{FALSE}\NormalTok{, }\AttributeTok{method =} \StringTok{"lm"}\NormalTok{)}
\end{Highlighting}
\end{Shaded}

\begin{verbatim}
## `geom_smooth()` using formula = 'y ~ x'
\end{verbatim}

\includegraphics{Analysis_files/figure-latex/unnamed-chunk-5-1.pdf}

\begin{Shaded}
\begin{Highlighting}[]
\NormalTok{mergedf }\SpecialCharTok{\%\textgreater{}\%}
  \FunctionTok{ggplot}\NormalTok{(}\FunctionTok{aes}\NormalTok{(}\AttributeTok{x =}\NormalTok{ admission, }\AttributeTok{y =}\NormalTok{ population\_2020)) }\SpecialCharTok{+}
  \FunctionTok{geom\_point}\NormalTok{() }\SpecialCharTok{+}
  \FunctionTok{geom\_smooth}\NormalTok{(}\AttributeTok{se =} \ConstantTok{FALSE}\NormalTok{, }\AttributeTok{method =} \StringTok{"lm"}\NormalTok{)}
\end{Highlighting}
\end{Shaded}

\begin{verbatim}
## `geom_smooth()` using formula = 'y ~ x'
\end{verbatim}

\includegraphics{Analysis_files/figure-latex/unnamed-chunk-5-2.pdf}

\begin{Shaded}
\begin{Highlighting}[]
\NormalTok{namedPop }\OtherTok{=} \FunctionTok{lm}\NormalTok{(}\AttributeTok{data =}\NormalTok{ mergedf, }\AttributeTok{formula =}\NormalTok{ population\_2020 }\SpecialCharTok{\textasciitilde{}}\NormalTok{ admission)}
\FunctionTok{summary}\NormalTok{(namedPop)}
\end{Highlighting}
\end{Shaded}

\begin{verbatim}
## 
## Call:
## lm(formula = population_2020 ~ admission, data = mergedf)
## 
## Residuals:
##      Min       1Q   Median       3Q      Max 
## -6921747 -3808062 -1652801  1154594 33586073 
## 
## Coefficients:
##               Estimate Std. Error t value Pr(>|t|)
## (Intercept) 2719234.81 2736424.64   0.994    0.325
## admission       -74.19      54.80  -1.354    0.181
## 
## Residual standard error: 7106000 on 54 degrees of freedom
## Multiple R-squared:  0.03283,    Adjusted R-squared:  0.01492 
## F-statistic: 1.833 on 1 and 54 DF,  p-value: 0.1814
\end{verbatim}

One interesting question could be the differences in state formation and
current population between states with native American origins vs
European origins which we will use the etymology of their state name

\end{document}
